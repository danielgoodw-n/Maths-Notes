\documentclass{article}

\usepackage[utf8]{inputenc}
\usepackage{csquotes}
\usepackage[english]{babel}
\usepackage{amsmath,amssymb,amsthm,textcomp}
\usepackage{mathtools}
\usepackage{biblatex}
\usepackage{tikz}
\usepackage{graphics, setspace}
\usepackage{listings}
\usepackage{lipsum}
\usepackage{hyperref}
\hypersetup{
    colorlinks,
    citecolor=black,
    filecolor=black,
    linkcolor=black,
    urlcolor=black
}

\DeclareMathAlphabet{\pazocal}{OMS}{zplm}{m}{n}
\DeclareMathOperator{\Ima}{Im}
\newcommand{\Ba}{\mathcal{B}}
\newcommand{\Ta}{\mathcal{T}}
\newcommand{\Aa}{\mathcal{A}}
\newcommand{\R}{\mathbb{R}}
\newcommand{\C}{\mathbb{C}}
\newcommand{\Z}{\mathbb{Z}}
\newcommand{\N}{\mathbb{N}}
\newcommand{\p}{\mathbb{P}}
\newcommand{\Ss}{\mathcal{S}} % Schwartz space
\newcommand{\F}{\mathcal{F}} % Fourier Transform
\newcommand{\Rf}{\mathcal{R}} % reflection
\newcommand{\E}{\mathbb{E}}

\DeclarePairedDelimiter\abs{\lvert}{\rvert}%
\DeclarePairedDelimiter\norm{\lVert}{\rVert}%
% Swap the definition of \abs* and \norm*, so that \abs
% and \norm resizes the size of the brackets, and the 
% starred version does not.
\makeatletter
\let\oldabs\abs
\def\abs{\@ifstar{\oldabs}{\oldabs*}}
%
\let\oldnorm\norm
\def\norm{\@ifstar{\oldnorm}{\oldnorm*}}
\makeatother

\newtheorem{theorem}{Theorem}[section]
\newtheorem{corollary}{Corollary}[theorem]
\newtheorem{lemma}[theorem]{Lemma}

\title{Complex Analysis - X\_400386}
\author{Yoav Eshel}
\date{\today}

\begin{document}
    \maketitle
    \tableofcontents
    \newpage

    This notes are based on Complex Variables and Applications by James Ward Brown (9th edition).
    \section{Regions in the Complex Plane}
    For $z_0\in\C$, its \textit{neighborhood} is the set
    $$
        \abs{z-z_0}\leq\varepsilon,\quad\varepsilon>0.
    $$
    It consists of all the point that lie inside (but not on!) the circle of radius $\varepsilon$ centered at $z_0$.
    A \textit{deleted neighborhood} is the same set as the neighborhood minus the origin, i.e
    $$
        0<\abs{z-z_0}<\varepsilon.
    $$
    A point $z_0$ is an \textit{interior point} of some set $S$, if there exists a $z_0$-neighborhood that contains only points in $S$.
    It is an \textit{exterior point} if there exists a neighborhood containing no points in $S$.
    If $z_0$ is neither, then it is a \textit{boundary point}.

    A set is \textit{open} if it does not contain any of its boundary points ($\iff$ every point is an interior point) and \textit{closed} if it contains all of them.
    A set can be neither or both. The disk $0<\abs{z}\leq 1$ is neither while the entire complex plane is both.

    An open set $S$ is \textit{connected} if there for any two points in the set there is a finite number of straight lines (polygonal line), all in $S$, that connect them.
    A nonempty open set that is connected is called a \textit{domain} (e.g. every neighborhood is a domain). A connected set that is not open is called a \textit{region}. 

    A set is \textit{bounded} if it can be contained in some circle, otherwise it is \textit{unbounded}.
    
    A point $z_0$ is an \textit{accumulation point} of a set $S$ if every deleted neighborhood of $z_0$ contains at least one point of $S$.
    For example, the point $0$ is an accumulation point of the set $S=\left\{\frac{1}{n}\mid n\in\N\right\}$.
    
    \section{Analytic Functions}
    We say that $f(z)\to w_0$ as $z\to z_0=(x_0,y_0)$ if for every $\varepsilon>0$ there exists $\delta>0$ s.t.
    $$
        0<\abs{z-z_0}<\delta\implies\abs{f(z)-w_0}<\varepsilon\vspace{0.5em}
    $$
    where $\abs{z-z_0}=\abs{x-x_0+(y-y_0)i}=\sqrt{(x-x_0)^2+(y-y_0)^2}$. If such $w_0$ exists then it is unique.

    If $f:\C\to\C$ is given by
    $$
        f(z)=f(x+yi)=u(x,y)+iv(x,y)
    $$
    for $u,v$ real valued and
    $$
        \lim_{(x,y)\to(x_0,y_0)} u(x,y)=u_0,\quad \lim_{(x,y)\to(x_0,y_0)} v(x,y)=v_0
    $$
    exist, then
    $$
        \lim_{(x,y)\to(x_0,y_0)} f(x+yi) = u_0+v_0i.
    $$
    Therefore the limit rules for real valued functions are easily applied to complex functions.

    






\end{document}