\documentclass{article}

\usepackage[utf8]{inputenc}
\usepackage{csquotes}
\usepackage[english]{babel}
\usepackage{amsmath,amssymb,amsthm,textcomp}
\usepackage{mathtools}
\usepackage{biblatex}
\usepackage{tikz}
\usepackage{graphics, setspace}
\usepackage{listings}
\usepackage{lipsum}
\usepackage{bookmark}
\usepackage{hyperref}
\hypersetup{
    colorlinks,
    citecolor=black,
    filecolor=black,
    linkcolor=black,
    urlcolor=black
}

\DeclareMathAlphabet{\pazocal}{OMS}{zplm}{m}{n}
\DeclareMathOperator{\Ima}{Im}
\newcommand{\Ba}{\mathcal{B}}
\newcommand{\Ta}{\mathcal{T}}
\newcommand{\Aa}{\mathcal{A}}
\newcommand{\R}{\mathbb{R}}
\newcommand{\C}{\mathbb{C}}
\newcommand{\Z}{\mathbb{Z}}
\newcommand{\N}{\mathbb{N}}
\newcommand{\Q}{\mathbb{Q}}
\newcommand{\p}{\mathbb{P}}
\newcommand{\Ss}{\mathcal{S}} % Schwartz space
\newcommand{\F}{\mathcal{F}} % Fourier Transform
\newcommand{\Rf}{\mathcal{R}} % reflection
\newcommand{\E}{\mathbb{E}}

\DeclarePairedDelimiter\abs{\lvert}{\rvert}%
\DeclarePairedDelimiter\norm{\lVert}{\rVert}%
% Swap the definition of \abs* and \norm*, so that \abs
% and \norm resizes the size of the brackets, and the 
% starred version does not.
\makeatletter
\let\oldabs\abs
\def\abs{\@ifstar{\oldabs}{\oldabs*}}
%
\let\oldnorm\norm
\def\norm{\@ifstar{\oldnorm}{\oldnorm*}}
\makeatother

\newtheorem{theorem}{Theorem}[section]
\newtheorem{corollary}{Corollary}[theorem]
\newtheorem{lemma}[theorem]{Lemma}
\newtheorem*{definition}{Definition}
\newtheorem*{remark}{Remark}

\theoremstyle{remark}
\newtheorem*{sol}{Solution}

\newenvironment{ex}[1]
    {\noindent\textbf{Exercise #1}\normalsize\newline}
    {\vspace{0.5 em}}

\title{Complex Analysis - X400386}
\author{Yoav Eshel}
\date{\today}

\begin{document}
    \maketitle
    \tableofcontents
    \newpage

    These notes are based on Complex Variables and Applications by James Ward Brown ($9^\text{th}$ edition).
    \section{Regions in the Complex Plane}
    For $z_0\in\C$, its \textit{neighborhood} is the set
    $$
        \abs{z-z_0}\leq\varepsilon,\quad\varepsilon>0.
    $$
    It consists of all the point that lie inside (but not on!) the circle of radius $\varepsilon$ centered at $z_0$.
    A \textit{deleted neighborhood} is the same set as the neighborhood minus the origin, i.e
    $$
        0<\abs{z-z_0}<\varepsilon.
    $$
    A point $z_0$ is an \textit{interior point} of some set $S$, if there exists a $z_0$-neighborhood that contains only points in $S$.
    It is an \textit{exterior point} if there exists a neighborhood containing no points in $S$.
    If $z_0$ is neither, then it is a \textit{boundary point}.

    A set is \textit{open} if it does not contain any of its boundary points ($\iff$ every point is an interior point) and \textit{closed} if it contains all of them.
    A set can be neither or both. The disk $0<\abs{z}\leq 1$ is neither while the entire complex plane is both.

    An open set $S$ is \textit{connected} if there for any two points in the set there is a finite number of straight lines (polygonal line), all in $S$, that connect them.
    A nonempty open set that is connected is called a \textit{domain} (e.g. every neighborhood is a domain). A connected set that is not open is called a \textit{region}. 

    A set is \textit{bounded} if it can be contained in some circle, otherwise it is \textit{unbounded}.
    
    A point $z_0$ is an \textit{accumulation point} of a set $S$ if every deleted neighborhood of $z_0$ contains at least one point of $S$ (i.e. we can get arbitrarily close to $z_0$ while staying in $S$).
    For example, the point $0$ is an accumulation point of the set $S=\left\{\frac{1}{n}\mid n\in\N\right\}$.
    
    \section{Analytic Functions}
    We say that $f(z)\to w_0$ as $z\to z_0=(x_0,y_0)$ if for every $\varepsilon>0$ there exists $\delta>0$ s.t.
    $$
        0<\abs{z-z_0}<\delta\implies\abs{f(z)-w_0}<\varepsilon\vspace{0.5em}
    $$
    where $\abs{z-z_0}=\abs{x-x_0+(y-y_0)i}=\sqrt{(x-x_0)^2+(y-y_0)^2}$. If such $w_0$ exists then it is unique.

    If $f:\C\to\C$ is given by
    $$
        f(z)=f(x+yi)=u(x,y)+iv(x,y)
    $$
    for $u,v$ real valued and
    $$
        \lim_{(x,y)\to(x_0,y_0)} u(x,y)=u_0,\quad \lim_{(x,y)\to(x_0,y_0)} v(x,y)=v_0
    $$
    exist, then
    $$
        \lim_{(x,y)\to(x_0,y_0)} f(x+yi) = u_0+v_0i.
    $$
    Therefore the limit rules for real valued functions are easily applied to complex functions.

    If we center a unit sphere on the origin of the complex plane, then every point $z$ on the plane is mapped to a unique point $P$ on the surface of the sphere, where $P$ is the intersection of the line from $z$ to the north pole of the sphere. 
    Note that points outside the unit circle are mapped to the northern hemisphere while points inside the unit circle are mapped to the southern hemisphere. Then the unit circle is mapped to the equator and the origin is mapped to the south pole.
    The \textit{point of infinity} is defined as the point that is mapped to the north pole. Then for small $\varepsilon$ the set $\abs{z}>\frac{1}{\varepsilon}$ is called a \textit{neighborhood of $\infty$}.
    Thus we can simply replace the neighborhoods in the definition of a limit by neighborhoods of $\infty$.

    The derivative in complex analysis is defined in the same manner as in usual calculus. In other words, if
    $$
    f'(z):=\lim_{h\to0}\frac{f(z+h)-f(z)}{h}
    $$
    exists then $f$ is differentiable at $z$.

    \subsection{Cauchy-Riemann Equations}
    Let $f:\C\to\C$ given by $f(x+yi)=u(x,y)+v(x,y)i$. Suppose $f$ is differentiable at some $z_0$ and $'f(z_0)=a+bi$.
    Then we can approximate $f$ using
    \begin{align*}
        f(z_0+h)&=f(z_0)+f'(z_0)\cdot h+\mathcal{O}(h)
    \end{align*}
    Let $h=x+yi$. Then
    \begin{align*}
        (a+bi)(x+yi)&=\begin{pmatrix}
            ax-by\\
            ay+bx\\
        \end{pmatrix}\\
        &=\begin{pmatrix}
            a&-b\\
            b&a\\
        \end{pmatrix}
        \begin{pmatrix}
            x\\ y \\
        \end{pmatrix}
    \end{align*}
    and it follows that the matrix $\begin{pmatrix}a&-b\\b&a   \end{pmatrix}$ is the Jacobian of $f$ viewed as a function from $\R^2$ to $\R^2$!
    Hence $\partial_x u = \partial_y v$ and $\partial_y u = -\partial_x v$. This is a necessary condition for a complex function to be differentiable.
    If the partial derivatives exist and are continuous around $z_0$ and  satisfy the relation above then $f'(z_0)$ exists.
    In other words, it means that a complex differentiable function is locally a rotation and a scaling.
    It follows that if $f$ is differentiable then
    $$
        f'(x+yi)=\partial_x u(x,y)+i\partial_x v(x,y)
    $$

    \newpage
    \section{Exercises}
    
\subsection{Symmetric Polynomial}
    \begin{ex}{14.10}
        Express the symmetric polynomials $\sum_n T_1^2T_2$ and $\sum_{n} T_1^3T_2$ in the elementary symmetric polynomials.
    \end{ex}
    \begin{sol}
        To get the polynomial $\sum_n T_1^2T_2$ we start with
        $$
            s_1s_2=\sum_n T_1\sum_n T_1T_2 = \sum_n T_1^2T_2+3\sum_n T_1T_2T_3 = \sum_n T_1^2T_2+3s_3
        $$
        Thus 
        $$
            \sum_n T_1^2T_2 = s_1s_2-3s_3
        $$

        Similarly, to transform the polynomial $\sum_{n} T_1^3T_2$ we start with
        \begin{align*}
            s_1^2s_2&=\left(\sum_nT_1\right)^2\sum_nT_1T_2\\
            &=\left(\sum_n T_1^2+2\sum_n T_1T_2\right)\sum_nT_1T_2\\
            &=\sum_nT_1^2\sum_n T_1T_2+2s_2^2\\
            &=\sum_nT_1^3T_2+\sum_n T_1^2T_2T_3+2s_2^2.
        \end{align*}
        And since
        $$
            s_1s_3=\sum_nT_1\sum_nT_1T_2T_3=\sum_nT_1^2T_2T_3+4\sum_n T_1T_2T_3T_4
        $$
        it follows that $\sum_n T_1^2T_2T_3=s_1s_3-4s_4$ and so
        $$
            \sum_{n} T_1^3T_2=s_1^2s_2-s_1s_3+4s_4-2s_2^2
        $$
    \end{sol}

    \begin{ex}{14.14}
        Prove: For $n\in\Z_{>0}$, we have $\Delta(X^n+a)=(-1)^{\frac12n(n-1)}n^na^{n-1}$.
    \end{ex}
    \begin{proof}
        Let $f(X)=X^n+a$ and let $\alpha_i$ be its roots. Then $f'(X)=nX^{n-1}$ and
        $$
            \Delta(f)=(-1)^{n(n-1)/2}R(f,f').
        $$
        Let $f_1(X)=a$ and then $f\equiv f_1\mod(f')$ since $f = f_1+f'\cdot\left(\frac1n X\right)$.
        Simplifying the resultant we get
        \begin{align*}
            R(f,f')&=R(f',f)&&(\text{Property }1)\\
            &=n^{n}R(f',f_1)&&(\text{Property }3)\\
            &=n^{n}\cdot\left(n^0\prod_{i=1}^{n-1}f_1(\alpha_i)\right)&&(\text{Property }2)\\
            &=n^n a^{n-1}
        \end{align*}
        and the result follows.
    \end{proof}

    \begin{ex}{14.15}
        Calculate the discriminant of the polynomial $f(X)=X^4+pX+q\in\Q(p,q)[X]$.
    \end{ex}
    \begin{sol}
        Then $f'(X)=4X^3+p$ and so 
        $$f_1(X)=f-f'\cdot h = X^4+pX+q+(4X^3+p)(\frac14 X) = \frac{3p}{4}X+q.$$
        Then the resultant is
        \begin{align*}
            R(f,f')&=R(f',f)&&(\text{Property } 1)\\
            &=4^{4-1}R(f', f_1)&&(\text{Property } 3)\\
            &=4^3\left((-1)^{3\cdot 1}R(f_1,f')\right)&&(\text{Property } 1)\\
            &=-4^3\left(\left(\frac{3p}{4}\right)^3\prod_{i=1}^{1}f'\left(\frac{-4q}{3p}\right)\right)&&(\text{Property } 2)\\
            &=-3^3p^3\left(4\left(\frac{-4q}{3p}\right)^3+p\right)\\
            &=4^4q^3-3^3p^4.
        \end{align*}
        Therefore the discriminant of $f$ is
        $$
            \Delta(f) = (-1)^{4\cdot 3/2}R(f,f') = R(f, f') = 4^4q^3-3^3p^4.
        $$
    \end{sol}

    \begin{ex}{14.16}
        For every $n>1$, determine an expression for the discriminant of the polynomial $f(X) = X^n+pX+q\in\Q(p,q)[X]$.
    \end{ex}
    \begin{sol}
        Let $f(X)=X^n+pX+q\in\Q(p,q)[X]$ for $n>1$. 
        Then $f'(X)=nX^{n-1}+p$ and $f\equiv f_1\mod(f')$ where
        $$f_1 = f-f'\cdot h = X^n+pX+q-\left(nX^{n-1}+p\right)\left(\frac1n X\right)=\frac{p(n-1)}{n}X+q.$$
        The resultant of $f$ and $f'$ is given by
        \begin{align*}
            R(f,f') &= R(f', f)&&(\text{Property } 1)\\
            &=n^{n-1}R(f', f_1)&&(\text{Property } 3)\\
            &=n^{n-1}\left((-1)^{n-1}R(f_1, f')\right)&&(\text{Property } 1)\\
            &=(-n)^{n-1}\left(\frac{p(n-1)}{n}\right)^{n-1}\prod_{i=1}^1 f'\left(-\frac{nq}{(n-1)p}\right)&&(\text{Property } 2)\\
            &=(-1)^{n-1}p^{n-1}(n-1)^{n-1}\left(\frac{(-1)^{n-1}n^nq^{n-1}}{(n-1)^{n-1}p^{n-1}}+p\right)\\
            &=n^nq^{n-1}+(-1)^{n-1}p^n(n-1)^{n-1}.
        \end{align*}
        Hence the discriminant of $f$ is
        $$
            \Delta(f)=(-1)^{n(n-1)/2}R(f,f')=(-1)^{n(n-1)/2}\left(n^nq^{n-1}+(-1)^{n-1}p^n(n-1)^{n-1}\right)
        $$
    \end{sol}

    \begin{ex}{14.17}
        Let $f\in\Z[X]$ be a monic polynomial. Prove that the following are equivalent
        \begin{enumerate}
            \item $\Delta(f)\neq 0$.
            \item The polynomial $f$ has no double zeroes in $\C$.
            \item The decomposition of $f$ in $\Q[X]$ has no multiple prime factors.
            \item The polynomial $f$ and its derivative $f'$ are relatively prime in $\Q[X]$.
            \item The polynomial $f\mod p $ and $f' \mod p$ are relatively prime in $\mathbb{F}_p[X]$ for almost all prime numbers $p$.
        \end{enumerate}
    \end{ex}
    \begin{proof}
        Let  $f\in\Z[X]$ be monic and $\{\alpha_1,\alpha_2,\dots,\alpha_n\}$ it roots in $\C$.

        $(1)\Rightarrow  (2)$. Suppose that $\alpha_i=\alpha_j$ for some $i\neq j$. Then 
        $$\Delta(f)=\prod_{1\leq i<j\leq n}(\alpha_i-\alpha_j)= 0,$$ 
        which is a contradiction.
        Therefore if $f$ has non-zero discriminant it has no double zeroes in $\C$. 

        $(2)\Rightarrow (3)$.
        
        $(3)\Rightarrow (4)$.

        $(4)\Rightarrow (5)$. If $f$ and $f'$ are relatively prime in $\Q[X]$ then 

        $(1)\Rightarrow (1)$. 
    \end{proof}

    \begin{ex}{14.19}
        Let $f\in\Q[X]$ be a monic polynomial with $n=\deg(f)$ distinct complex roots. Prove: the sign of $\Delta(f)$ is equal to $(-1)^s$ where $2s$ is the number of non-real zeroes of $f$.
    \end{ex}
    \begin{proof}
        Let $\{\alpha_1,\dots,\alpha_{n}\}$ be all the roots of $f$.
        Then each term $(\alpha_i-\alpha_j)^2$ in the discriminant falls into one of 3 cases
        \begin{enumerate}
            \item Both $\alpha_i$ and $\alpha_j$ are non-real. Then
            \begin{enumerate}
                \item If $\alpha_j=\overline{\alpha_i}$ then $\alpha_i-\alpha_j$ is purely complex and $(\alpha_i-\alpha_j)^2$ is negative.
                \item If $\alpha_j\neq\overline{\alpha_i}$ then $\overline{\alpha_i}$ and $\overline{\alpha_j}$ are also roots of $f$ and the term
                $$(\alpha_i-\alpha_j)^2(\overline{\alpha_i}-\overline{\alpha_j})^2=\left((\overline{\alpha_i-\alpha_j})(\alpha_i-\alpha_j)\right)^2=\abs{\alpha_i-\alpha_j}^2 $$
                is positive.
            \end{enumerate}
            \item $\alpha_i$ is non-real and $\alpha_j$ is real. Then $\overline{\alpha_i}$ is a root of $f$ and the term
            $$(\alpha_i-\alpha_j)^2(\overline{\alpha_i}-\alpha_j)^2=\abs{\alpha_i-\alpha_j}^2 $$
            is positive.
            \item Both $\alpha_i$ and $\alpha_j$ are real. Then $(\alpha_i-\alpha_j)^2$ is positive.
        \end{enumerate}
        Since the only negative terms are of the form $(\alpha_i-\overline{\alpha_i})^2$ and there are $2s$ non-real roots the sign of the determinant is $(-1)^s$.

    \end{proof}

    \begin{ex}{14.20}
        Prove: $f(X)=X^3+pX+q\in\R[X]$ has three (counted with multiplicity) real zeroes $\iff$ $4p^3+27q^\leq 0$.
    \end{ex}
    \begin{proof}
        By Ex. 16 we know that $\Delta(f)=(-1)^3\left(3^3q^2+2^2p^3\right)=-27q^2-4p^3$. 
        Let $a,b$ and $c$ be the roots of $f$. If $a,b,c\in\R$ then 
        $$
        -27q^2-4p^3=\Delta(f)=(a-b)^2(a-c)^2(b-c)^2\geq 0
        $$
        and so $4p^3+27q^\leq 0$.

        Now suppose that $a=x+yi$ and $b=x-yi$ are complex conjugates and $c$ is real. Then 
        \begin{align*}
            -27q^2-4p^3&=\Delta(f)\\
            &=(a-b)^2(a-c)^2(b-c)^2\\
            &=-4y^2\left((a-c)(\overline{a-c})\right)^2\\
            &=-4y^2\abs{a-c}^2\\
            &\leq 0.
        \end{align*}
        Hence $4p^3+27q^\geq 0$ and the result follows by contraposition.
    \end{proof}
        
    \begin{ex}{14.21}
        Express $p_4=\sum_nT_1^4$ in elementary symmetric polynomials
    \end{ex}
    \begin{sol}
        Let $n\geq 4$. Starting with
        \begin{align*}
            s_1^4 &= \left(\sum_nT_1\right)^4\\& = \sum_n T_1^4+4\sum_n T_1^3T_2+12\sum_n T_1^2T_2T_3+6\sum_nT_1^2T_2^2+24\sum_nT_1T_2T_3T_4.
        \end{align*}
        To understand how to coefficients of the sum are obtained, consider the number of ways the $T_i$ can be arranged. 
        For example, $T_1^4=T_1T_1T_1T_1$ can only be arranged in 1 way but $T_1^2T_2T_3=T_1T_1T_2T_3$ can be arrange in $\frac{4!}{2}=12$ ways (where we divided by 2 since the two $T_1$ can be swapped in any given arrangement).
        Then
        $$
            s_1^2s_2=\left(\sum_n T_1\right)^2s_2=\left(\sum_nT_1^2+2\sum_n T_1T_2\right)s_2 = \sum_n T_1^3T_2+\sum_nT_1^2T_2T_3+2s_2^2.
        $$
        So far we have
        \begin{align*}
            p_4 &= s_1^4-4\left(s_1^2s_2-2s_2^2-\sum_nT_1^2T_2T_3\right)-12\sum_n T_1^2T_2T_3-6\sum_nT_1^2T_2^2-24\sum_nT_1T_2T_3T_4\\
            &=s_1^4-4s_1^2s_2+8s_2^2-24s_4-6\sum_nT_1^2T_2^2-8\sum_n T_1^2T_2T_3.
        \end{align*}
        So continuing with $\sum_nT_1^2T_2^2$ we get
        $$
            s_2^2 = \left(\sum_n T_1T_2\right)^2=\sum_n T_1^2T_2^2+2\sum_n T_1^2 T_2T_3+6\sum_n T_1T_2T_3T_4.
        $$
        Finding the coefficients here is slightly trickier since $s_2$ contains pairs not all arrangements are allowed. 
        For example, $T_1^2T_2^2$ can only come from the pair $T_1T_2$. On the other hand $T_1T_2T_3T_4$ can come from $T_1T_2$ and $T_3T_4$ or $T_1T_4$ and $T_2T_3$ and so on.
        We choose the first pair (${4\choose 2}=6$ ways) which also fixes the second pair and so there are 6 ways to get $T_1T_2T_3T_4$.
        Hence
        \begin{align*}
            p_4 &= s_1^4-4s_1^2s_2+8s_2^2-24s_4-6\left(s_2^2-2\sum_nT_1^2T_2T_3-6s_4\right)-8\sum_n T_1^2T_2T_3\\
            &=s_1^4-4s_1^2s_2+2s_2^2+12s_4+4\sum_n T_1^2T_2T_3.
        \end{align*}
        Using Exercise 14.10 we get
        \begin{align*}
            p_4 &=s_1^4-4s_1^2s_2+2s_2^2+12s_4+4(s_1s_3-4s_4)\\
            &=s_1^4-4s_1^2s_2+2s_2^2-4s_4+4s_1s_3
        \end{align*}
    \end{sol}

    \begin{ex}{14.22}
        A rational function $f\in\Q[T_1,\dots,T_n]$ is called symmetric if it is invariant under all permutations of the variables $T_i$. Prove that every symmetric rational function is a rational function in the elementary symmetric functions.
    \end{ex}
    \begin{proof}
        Let $f\in\Q[T_1,\dots,T_n]$ be a symmetric rational function. 
        Then $f=g/h$ for $g,h$ polynomials. If $h$ is a symmetric polynomial then $g=fh$ is symmetric as well.
        By the fundamental theorem of symmetric polynomial both $g$ and $h$ can be written in terms of elementary symmetric polynomials and we're done.
        If $h$ is not symmetric, then let 
        $$\tilde{h}=\prod_{\sigma\in S_n\setminus\{e\}}\sigma(h)$$
        and then $h\tilde{h}$ is symmetric so $f=\frac{g\tilde{h}}{h\tilde{h}}$ which is again the case above.
    \end{proof}

    \begin{ex}{14.23}
        Write $\sum_{n}T_1^{-1}$ and $\sum_n T_1^{-2}$ as rational functions in $\Q[s_1,\dots,s_n]$
    \end{ex}
    \begin{sol}
        Starting with
        $$
            \sum_{n}T_1^{-1}=\frac{1}{T_1}+\cdots+\frac{1}{T_n}.
        $$
        We multiply by $1=\frac{s_n}{s_n}$ and simplify
        \begin{align*}
            \frac{s_n}{s_n}\sum_{n}T_1^{-1}&=\frac{T_1T_2\cdots T_n}{T_1T_2\cdots T_n}\left(\frac{1}{T_1}+\cdots+\frac{1}{T_n}\right)\\
            &=\frac{s_{n-1}}{s_n}
        \end{align*}

        For the second expression we present to approaches.
        \begin{enumerate}
            \item Observing that 
                $$\left(\sum_n T_1^{-1}\right)^2=\sum_{n} T_1^{-2}+2\sum_{n}T_1^{-1}T_2^{-1}$$
            we can write using the previous part
                $$ \sum_n T_1^{-2} = \frac{s_{n-1}^2}{s_n^2}-2\sum_{n}T_1^{-1}T_2^{-1}$$
            and multiplying by the second term by $\frac{s_{n}}{s_{n}}$ we get
                $$ \sum_n T_1^{-2} = \frac{s_{n-1}^2}{s_n^2} - 2\left(\frac{1}{T_1T_2}+\cdots+\frac{1}{T_{n-1}T_n}\right)\frac{T_1\cdots T_n}{T_1\cdots T_n}=\frac{s_{n-1}^2}{s_n^2} - 2\frac{s_{n-2}}{s_n}.$$
            Hence $\sum_n T_1^{-2}=\frac{s_{n-1}^2-2s_{n-2}s_n}{s_n^2}$.
            \item The second approach is slightly more involved. We start by multiplying by 1 in a clever (but different) way
                $$\left(\sum_n T_1^{-2}\right)\frac{s_n^2}{s_n^2}=\left(\frac{1}{T_1^2}+\cdots+\frac{1}{T_n^2}\right)\frac{T_1^2\cdots T_n^2}{T_1^2\cdots T_n^2}=\frac{\sum_n T_1^2\cdots T_{n-1}^2}{s_n^2}.$$
            Then $\sum_n T_1^2\cdots T_{n-1}^2$ is obviously (condescending much?) a symmetric polynomial and so we can use our trusty algorithm. Starting with
            \begin{align*}
                s_1^{2-2}s_2^{2-2}\cdots s_{n-1}^{2-0}&=s_{n-1}^2\\
                &=\left(\sum_n T_1\cdots T_{n-1}\right)^2\\
                &=\sum_n T_1^2\cdots T_{n-1}^2 + 2\sum_n T_1^2\cdots T_{n-2}^2T_{n-1}T_n.
            \end{align*}
            Moving to the second term
            \begin{align*}
                s_1^{2-2}\cdots s_{n-2}^{2-1}s_{n-1}^{1-1}s_n^1&=s_{n-2}s_n\\
                &=\left(\sum_n T_1\cdots T_{n-2}\right)T_1\cdots T_n\\
                &=\sum_n T_1^2\cdots T_{n-2}^2 T_{n-1}T_n
            \end{align*}
            and it follows that
            $$\sum_n T_1^2\cdots T_{n-1}^2 = s_{n-1}^2-2s_{n-2}s_n.$$
            So we conclude that
            $$ \sum_n T_1^{-2} = \frac{s_{n-1}^2-2s_{n-2}s_n}{s_n^2}$$
            which is reassuring.
        \end{enumerate}
        Note that in the first approach we stumbled upon something rather interesting:
        $$
            \sum_n T_1^{-1}\cdots T_k^{-1} = \frac{s_{n-k}}{s_n}
        $$
        the proof of which is left as an exercise to the reader.
    \end{sol}

    \begin{ex}{14.24}
        
    \end{ex}

\subsection{Field Extensions}
    \begin{ex}{21.18}
        Let $K\subset L$ be an algebraic extension. For $\alpha, \beta\in L$ prove that we have
        $$ \left[K(\alpha,\beta):K\right]\leq\left[K(\alpha):K\right]\cdot\left[K(\beta):K\right]$$.
    \end{ex}
    \begin{proof}
        Let $f$ and $g$ be the minimal polynomials of $\al$ and $\be$ (respectively) in $K[x]$ and $f'$ be the minimal polynomial of $\alpha$ in $K(\beta)[x]$.
        If $\deg f'> \deg f$ then $f$ is a lower degree polynomial in $K(\beta)[x]$ with $f(\alpha)=0$ which is a contradiction. Hence $\deg f'\leq \deg f$ and so
        \begin{align*}
            \left[K(\alpha,\beta):K\right]&=\left[K(\al, \be):K(\be)\right]\cdot\left[K(\be):K\right]\\
            &=\deg f'\cdot \deg g\\
            &\leq\deg f\cdot \deg g\\
            &=\left[K(\al):K\right]\cdot \left[K(\be):K\right],   
        \end{align*}
        as desired.
    \end{proof}

    \begin{ex}{21.19}
        Let $K\subset K(\al)$ be an extension of odd degree. Prove that $K(\al^2)=K(\al)$.
    \end{ex}
    \begin{proof}
        Let $f$ be the minimal polynomial of $\al$ in $K[x]$. Then $\deg f=2n+1$ for some $n\in\Z_+$. 
        Since $\al^2\in K(\al)$ we get the tower $K(\al)/K\left(\al^2\right)/K$ and so\
        $$ \left[K(\al):K\right]=\left[K(\al):K\left(\al^2\right)\right]\cdot\left[K\left(\al^2\right):K\right].$$
        Let $g$ be the minimal polynomial of $\al$ in $K\left(\alpha^2\right)$. Then $\deg g\leq 2$ since $x^2-\al^2\in K\left(\alpha^2\right)$ is a polynomial with a root $\al$.
        Since $\left[K(\al):K\right]$ is odd, it is not divisible by two and so $\deg g = 1$. Hence $\left[K(\al):K\left(\al^2\right)\right]=1$ and it follows that $K(\al)=K\left(\al^2\right)$.
    \end{proof}

    \begin{ex}{21.23}
        Show that every quadratic extension of $\Q$ is of the form $\Q\left(\sqrt{d}\right)$ with $d\in\Z$.
        For what $d$ do we obtain the cyclotomic field $\Q(\zeta_3)$?
    \end{ex}
    \begin{proof}
        Let $K/\Q$ be a quadratic extension. Take $\al\in K\setminus\Q$. Then 
        $$ \Q\subset\Q(\al)\subset K $$
        and so
        $$2=\left[K:\Q\right]=\left[K:\Q(\al)\right]\left[\Q(\al):\Q\right].$$
        If $\left[\Q(\al):\Q\right]=1$ then $\Q(\al)=\Q$ and so $\al\in\Q$, which contradicts our assumption. 
        It follows that $\left[K:\Q(\al)\right]=1$ and so $K=\Q(\al)$. 
        Let 
        $$f(x)=x^2+a_1 x+a_0\in\Q[x]$$
        be the minimal polynomial of $\al$. 
        Let $d=\frac{a_1^2}{4}-a_0\in\Q$ and note that $a_0=-\al a_1-\al^2$. Then
        \begin{align*}
            \sqrt{d}&=\sqrt{\frac{a_1^2}{4}-a_0}\\
            &=\sqrt{\frac{a_1^2}{4}+a_1\al+\al^2}\\
            &=\frac{a_1+2\al}{2}.
        \end{align*}
        Hence $\sqrt{d}\in\Q(\al)$. By similar calculations we get $\al=\frac{2\sqrt{d}-a_1}{2}\in\Q(\sqrt{d})$.
        Hence $K=\Q(\al)=\Q(\sqrt{d})$. Of course, it is not yet the case the $d$ is an integer.
        Suppose that $d=\frac{p}{q}$. Since $\sqrt{d}=\frac{1}{q^2}\sqrt{qp}\in\Q(\sqrt{qp})$ we have
        $$K=\Q(\al)=\Q(\sqrt{d})=\Q(\sqrt{qp})$$
        with $qp\in\Z$ as desired.
    \end{proof}

    \begin{ex}{21.24}
        Is every cubic extension of $\Q$ of the form $\Q\left(\sqrt[3]{d}\right)$ for some $d\in\Q$?
    \end{ex}
    \begin{sol}
        No.
    \end{sol}

    \begin{ex}{21.26}
        Let $M=\Q(\al)=\Q(1+\sqrt{2}+\sqrt{3})$. Show that $M$ is of degree 4 over $\Q$, determine the minimal polynomial and write $\sqrt{2}$ and $\sqrt{3}$ in the basis $\{1,\al, \al^2,\al^3\}$.
        Also prove that the group $G=\text{Aut}_\Q(M)$ is isomorphic to $V_4$ and that $f^\al_\Q=\prod_{\sigma\in G}X-\sigma(\al)\in\Q[X]$.
    \end{ex}
    \begin{sol}
        Let $\be = \al-1=\sqrt{2}+\sqrt{3}$. Then clearly $M=\Q(\al)=\Q(\be)$. Let
        \begin{align*}
            f(x)&=(x-\sqrt{2}-\sqrt{3})(x+\sqrt{2}-\sqrt{3})(x-\sqrt{2}+\sqrt{3})(x+\sqrt{2}+\sqrt{3})\\
            &=x^4-10x^2+1\in\Q[x]
        \end{align*}
        and so $f(\be)=0$ by construction. 
        
        Is $f$ the minimal polynomial of $\be$ in $\Q[x]$? It is if we can prove that $[M:\Q]=4$.
        From
        $$ (\sqrt{2}+\sqrt{3})(\sqrt{3}-\sqrt{2})=1 $$
        It follows that $\be^{-1}=\sqrt{3}-\sqrt{2}$. Therefore
        $$ \sqrt{2}=\frac12(\be-\be^{-1})\quad\text{and}\quad\sqrt{3}=\frac12(\be+\be^{-1})$$
        and so $M=\Q(\sqrt{2}+\sqrt{3})=\Q(\sqrt{2},\sqrt{3})$. 
        Hence we have the towers $M/\Q(\sqrt{2})/Q$ and $M/\Q(\sqrt{3})/Q$. 
        Let $g(x)=x^2-3$. Suppose it is not the minimal polynomial of $\sqrt{3}$ in $\Q(\sqrt{2})$.
        Then there exists $a+b\sqrt{2}\in\Q(\sqrt{2})$ such that
        $$ 0 = g(a+b\sqrt{2})=a^2+2b^2-3+2ab\sqrt{2}.$$
        But since
        \begin{equation*}
            \begin{cases}
                a^2+2b^2-3=0\\
                2ab=0
            \end{cases}
        \end{equation*}
        has no solutions it follows that no such element exists.
        Therefore $g$ is the minimal polynomial of $\sqrt{3}$ and $[M:\Q(\sqrt{2})]=\deg g=2$.
        Since $x^2-2$ is the minimal polynomial of $\sqrt{2}$ in $\Q$ we conclude that 
        $$[M:\Q]=[M:\Q(\sqrt{2})]\cdot[\Q(\sqrt{2}):\Q)]=4$$ 
        and therefore $f$ is the minimal polynomial of $\be$.

        Thus $f(x-1)$ is the minimal polynomial of $\al$ in $\Q$. 
        From $f(\be)=0$ it follows that $1=\beta(10\beta-\beta^3)$ and so $\be^{-1}=10\beta-\beta^3$.
        Hence
        $$\sqrt{2}=\frac12\left(\be-\be^{-1}\right)=\frac12\left(\be-10\be+\be^3\right)=\frac12\left(-9(\al-1)+(\al-1)^3\right)$$
        and
        $$\sqrt{3}=\frac12\left(\be+\be^{-1}\right)=\frac12\left(11(\al-1)-(\al-1)^3\right)$$

        Let $G=\text{Aut}(M)$ and take $\sigma\in G$. Then by definition $\sigma(1)=1$ and it follows by induction and the properties of isomorphism that $\sigma(a)=a$ for all $a\in\Z$.
        Since $1=\sigma(1)=\sigma(a\cdot a^{-1})=\sigma(a)\cdot\sigma(a)^{-1}=a\cdot a^{-1}$ it also follows that $\sigma\left(\frac{p}{q}\right)=\frac{p}{q}$. 
        Hence $\sigma$ restricted to $\Q$ is simply the identity map. 
        Therefore $\sigma$ is completely determined by $\sigma(\sqrt{2})$ and $\sigma(\sqrt{3})$.
        Since $0=\sigma(0)=\sigma(\sqrt{2}^2-2)=\sigma(\sqrt{2})^2-2$ the only options are $\sigma(\sqrt{2})=\pm\sqrt{2}$.
        Similarly we conclude that $\sigma(\sqrt{3})=\pm\sqrt{3}$. This gives four possible automorphism.
        Take $\sigma,\tau\in G$ such that $\sigma(\sqrt{2})=-\sqrt{2}, \sigma(x)=x$ $\forall x\in M\setminus\{\sqrt{2}\}$ and $\tau(\sqrt{3})=-\sqrt{3},\tau(x)=x$ $\forall x\in M\setminus\{\sqrt{3}\}$. 
        Since 
        $$\sigma\circ\sigma=\tau\circ\tau=\sigma\circ\tau\circ\sigma\circ\tau=e$$
        where $e$ is the identity map it follows that $G$ is isomorphic to $V_4$, the Klein four-group.

        Lastly, consider
        \begin{align*}
            \tilde{f}&=\prod_{\sigma\in G}x-\sigma(\al)\\
            &=(x-1-\sqrt{2}-\sqrt{3})(x-1+\sqrt{2}-\sqrt{3})(x-1-\sqrt{2}+\sqrt{3})\\&\qquad\qquad (x-1+\sqrt{2}+\sqrt{3}).
        \end{align*}
        Hence $\tilde{f}(x)=f(x-1)$ which we already proved is the minimal polynomial of $\al$ in $\Q[x]$.

    \end{sol}

    \begin{ex}{21.29}
        Take $K=\Q(\al)$ with $f^\al_\Q=x^3+2x^2+1$.
        \begin{enumerate}
            \item Determine the inverse of $\al+1$ in the basis $\{1,\al,\al^2\}$ of $K$ over $\Q$.
            \item Determine the minimal polynomial of $\al^2$ over $\Q$.
        \end{enumerate}
    \end{ex}

    \begin{ex}{21.30}
        Define the cyclotomic field $\Q(\zeta_3)$ and write $\al=\zeta_5^2+\zeta_5^3$.
        \begin{enumerate}
            \item Show that $\Q(\al)$ is a quadratic extension of $\Q$ and determine $f^\al_\Q$.
            \item Prove: $\Q(\al)=\Q(\sqrt{5})$
            \item Prove: $\cos(2\pi/5)=\frac{\sqrt{5}-1}{4}$ and $\sin(2\pi/5)=\sqrt{\frac{5+\sqrt{5}}{8}}$
        \end{enumerate}
    \end{ex}
\subsection{Finite Fields}

\subsection{Separable and Normal Extensions}

\end{document}