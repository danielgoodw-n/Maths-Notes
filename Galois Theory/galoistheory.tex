\documentclass{article}

\usepackage[utf8]{inputenc}
\usepackage{csquotes}
\usepackage[english]{babel}
\usepackage{amsmath,amssymb,amsthm,textcomp}
\usepackage{mathtools}
\usepackage{biblatex}
\usepackage{tikz}
\usepackage{graphics, setspace}
\usepackage{listings}
\usepackage{lipsum}
\usepackage{bookmark}
\usepackage{hyperref}
\hypersetup{
    colorlinks,
    citecolor=black,
    filecolor=black,
    linkcolor=black,
    urlcolor=black
}

\DeclareMathAlphabet{\pazocal}{OMS}{zplm}{m}{n}
\DeclareMathOperator{\Ima}{Im}
\newcommand{\Ba}{\mathcal{B}}
\newcommand{\Ta}{\mathcal{T}}
\newcommand{\Aa}{\mathcal{A}}
\newcommand{\R}{\mathbb{R}}
\newcommand{\C}{\mathbb{C}}
\newcommand{\Z}{\mathbb{Z}}
\newcommand{\N}{\mathbb{N}}
\newcommand{\Q}{\mathbb{Q}}
\newcommand{\p}{\mathbb{P}}
\newcommand{\Ss}{\mathcal{S}} % Schwartz space
\newcommand{\F}{\mathcal{F}} % Fourier Transform
\newcommand{\Rf}{\mathcal{R}} % reflection
\newcommand{\E}{\mathbb{E}}
\newcommand{\al}{\alpha}
\newcommand{\be}{\beta}
\newcommand{\de}{\delta}
\newcommand{\e}{\varepsilon}

\DeclarePairedDelimiter\abs{\lvert}{\rvert}%
\DeclarePairedDelimiter\norm{\lVert}{\rVert}%
% Swap the definition of \abs* and \norm*, so that \abs
% and \norm resizes the size of the brackets, and the 
% starred version does not.
\makeatletter
\let\oldabs\abs
\def\abs{\@ifstar{\oldabs}{\oldabs*}}
%
\let\oldnorm\norm
\def\norm{\@ifstar{\oldnorm}{\oldnorm*}}
\makeatother

\newtheorem{theorem}{Theorem}[section]
\newtheorem{corollary}{Corollary}[theorem]
\newtheorem{lemma}[theorem]{Lemma}
\newtheorem*{definition}{Definition}
\newtheorem*{remark}{Remark}

\theoremstyle{remark}
\newtheorem*{sol}{Solution}

\newenvironment{ex}[1]
    {\noindent\textbf{Exercise #1}\normalsize\newline}
    {\vspace{0.5 em}}

\setcounter{secnumdepth}{1}

\title{Galois Theory - 5122GALO6Y}
\author{Yoav Eshel}
\date{\today}

\begin{document}
    \maketitle
    \tableofcontents
    \newpage
    \section{Introduction}
    Galois theory is about studying Polynomials with coefficients in a field ($\Q,\R,\C$ etc.).
    Let
    $$
        f(T) = T^n+\cdots+a_1T+a_0\in\Q[T].
    $$
    Then $f(T)$ splits completely in $\C[T]$ as
    $$
        f(T)=(T-\alpha_1)\cdots(T-\alpha_n)
    $$
    with $\alpha_1,\dots \alpha_n\in\C$ are the roots of $f$. Galois theory studies permutation of the the roots that preserve algebraic relations between these roots.
    The allowed permutation of the roots give rise to a group denoted $\text{Gal}(f)$.
    The following definition of a Galois group does not require any background knowledge but is not very useful in practice.

    \begin{definition}
        Let $\sigma:\C\to\C$ be a field automorphism and $\alpha\in\C$ a root of $F(T)\in\Q[T]$. Since $\sigma(1)=1$ it follows that $\sigma(n)=n$ for all integers and so $\sigma(a/b)=\sigma(a)/\sigma(b)=a/b$ is the identity on $\Q$.
        Then
        \begin{align*}
            f(\sigma(\alpha))&=\sigma(\alpha)^n+\cdots+a_1\sigma(\alpha)+a_0\\
            &=\sigma(f(\alpha))\\
            &=0.
        \end{align*}
        Then each automorphism $\sigma$ is a permutation of the roots which is precisely the Galois group of the polynomial $\text{Gal}(f)\subset S_n$.
        In other words we have a group action 
        $$
        \text{Aut}(\C)\times\{\alpha_1,\dots,\alpha_n\}\to\{\alpha_1,\dots,\alpha_n\}
        $$
        Then $\text{Gal}(f):=\text{Im}(\phi)$ where $\phi:\text{Aut}(\C)\to S_n$ mapping $\sigma\mapsto(\alpha_i\mapsto \sigma(\alpha_i))$
    \end{definition}

    $\text{Gal}(f)\subset S_n$ is transitive subgroup (i.e. if its action on the set of roots is transitive) if and only if $f$ is irreducible.
    
    \section{Symmetric Polynomials}
    A symmetric polynomial is a polynomial $F(X_1,X_2,\dots, X_n)$ the is invariant under permutations of its variables.
    In other words
    $$ P(X_1, X_2,\dots, X_n)=P(X_{\sigma(1)}, X_{\sigma(2)},\dots,X_{\sigma(n)})$$
    for all $\sigma\in S_n$.
    Symmetric polynomials arise naturally in the study of the relation between the roots of a polynomial in one variable and its coefficients, 
    since the coefficients can be given by polynomial expressions in the roots, and all roots play a similar role in this setting.
    Let $f\in K(T)$ be a monic polynomial of degree $n$ that splits completely in $K$.
    Then
    $$ f(T)=(T-X_1)(T-X_2)\cdots(T-X_n)$$
    where $X_i$ are the roots of $f$. Then
    $$ f(T)=T^n+s_1 T^{n-1}+\cdots+(-1)^n s_n$$
    where
    \begin{align*}
        s_1&=X_1+X_2+\cdots+X_n\\
        s_2&=X_1X_2+X_1X_3+\cdots+X_{n-1}X_n\\
        \vdots&\\
        s_n&=X_1X_2\cdots X_n
    \end{align*}
    are called the \textit{elementary symmetric polynomials} in $X_1,X_2,\dot, X_n$.
    Then the fundamental theorem of symmetric polynomials states that every symmetric polynomial can be written as a polynomial expression in the elementary symmetric polynomials.

    To actually write a symmetric polynomial in terms of elementary symmetric polynomials we introduce some useful notation.
    We say a polynomial is ordered \textit{lexicographically} if the monomial $T_1^{e_1}T_2^{e_2}\cdots T_n^{e_n}$ with the highest $e_1$ is in front.
    If two monomials have the same $e_1$, then we compare their $e_2$ and so on. Like a dictionary.
    If $P$ is a symmetric polynomial in $n$ variables, choose a single representative proceeded by the symbol $\sum_n$ to denote the sum over the monomials in the $S_n$ orbit of the representative.
    Then for example
    \begin{align*}
        s_1&=\sum_n T_1\\
        s_2&=\sum_n T_1T_2\\
        \vdots&\\
        s_n&=\sum_n T_1T_2\cdots T_n=T_1T_2\cdots T_n.
    \end{align*}
    Now suppose $P$ is a symmetric polynomial. To find its representation in terms of symmetric polynomials:
    \begin{enumerate}
        \item Let $a\cdot T_1^{e_1}T_2^{e_2}\cdots T_n^{e_n}$ be the the first term in $P$, lexicographically.
        \item Form the monomial $$M = s_1^{e_1-e_2}s_2^{e_2-e_1}\cdots s_{n-1}^{e_{n-1}-e_n}s_n^{e_n}$$
        \item Let $P_i = P -cM$.
        \item Repeat steps (1)-(3) until $\deg P_i = 0$.
        \item The we can solve for $P$ and write it as a polynomial in the elementary symmetric polynomials.
    \end{enumerate}
    The representation obtained through the algorithm above is unique.

    The following theorem is useful when applying the algorithm above.
    \begin{theorem}[Orbit Stabilizer Theorem]\label{th:orbit_stabilizer}
        Let $G$ be a group acting on set $S$. For any $x\in S$ let $G_x=\{g\in G\mid g\cdot x= x\}$ denote the stabilizer of $x$, and let $G\cdot x=\{g\cdot x\mid g\in G\}$ denote the orbit of $x$. Then
        $$
            \abs{G}=\abs{G\cdot x}\abs{G_x}
        $$
    \end{theorem}
    Wondering how it might be useful? Consider 
    $$s_1^4=\left(\sum_n T_1\right)^4=(T_1+\cdots+T_n)(T_1+\cdots+T_n)(T_1+\cdots+T_n)(T_1+\cdots+T_n).$$ 
    After some thinking you might conclude that there are five possible representatives:
    $$T_1^4,\quad T_1^3T_2,\quad T_1^2T_2^2,\quad T_1^2T_2T_3\quad\text{and}\quad T_1T_2T_3T_4$$
    (note the the degrees always add up to four). But what are the coefficients? That's when the orbit-stabilizer theorem comes to the rescue.
    Let the permutation group $S_4$ act on the set of indices by permuting them. Then the coefficients in front of $\sum_n T_1^4$ is the size of the orbit of $(1,1,1,1)$.
    Since every permutation in $S_4$ return the same sequence, the size of the orbit is $\frac{4!}{4!}=1$. 
    Then the coefficients in front of $\sum_n T_1^2T_2^2$ is the size of the orbit of $(1,1,2,2)$.
    The permutations that fix it are $(1), (12), (34)$ and $(12)(34)$. So the size of the stabilizer is 4 and the size of the orbit is $\frac{4!}{4}=6$
    Similarly, the coefficients in front of $\sum_n T_1^2T_2T_3$ is the size of the orbit of $(1,1,2,3)$.
    Since the stabilizer contains only the permutations that switches the 1s and fixes the other two elements (namely $(1)$ and $(12)$) the size of the orbit is $ \frac{4!}{2}=12$.
    Lastly, the size of the orbit of $\sum_n T_1T_2T_3T_4$ is the $4!$ since there is no permutations (except the identity of course) that stabilizes it. We conclude that
    $$s_1^4=\sum_n T_1^4+6\sum_n T_1^2T_2^2+12\sum_n T_1^2T_2T_3+24\sum_n T_1T_2T_3T_4$$
    

    
    \section{Field Extensions}
    \subsection{Prime Fields}
    \begin{definition}
        Let $k$ be a field. Then the \textbf{prime field} in $K$ is the intersection over all subfields of $K$
    \end{definition}
    \begin{lemma}
        Let $K$ be a field of characteristic $k$. Then the prime field of $K$ is $\mathbb{F}_p$ if $k=p$ and $\Q$ if $k=0$.
    \end{lemma}
    \subsection{Algebraic and Transcendental Extensions}
    Let $L/K$ be a field extensions. Then we say that $\alpha\in L $ is \textit{algebraic} over $K$ if there exists an $f\in K[x], f\neq 0$, such that $f(\alpha)=0$.
    We say that $\alpha$ is \textit{transcendental} over $K$ if there exists no such $f$.
    The number of algebraic elements over $\Q$ in $\C$ is countable, so in fact $\C$ is mostly transcendental elements.

    \begin{theorem}
        Let $L/K$ be a field extension and take $\alpha\in L$. Then
        \begin{enumerate}
            \item If $\alpha$ is transcendental over k, then $K[\alpha]\simeq K[X]$
            \item If $\alpha$ is algebraic over $K$ then there exists $f\in K[X]$ monic and irreducible and
                $$ K[X]/f\simeq K[\alpha]=K(\alpha)$$
                and the degree of $L$ over $K$ is the degree of $f$.
        \end{enumerate}
    \end{theorem}

    \begin{definition}
        We say that an extension $L/K$ is \textbf{algebraic} if $\forall \alpha\in L$, $\alpha$ is algebraic over $K$.
    \end{definition}
    \begin{lemma}
        If a field extension is finite then it is algebraic.
    \end{lemma}
    The converse of this lemma does not hold.

    \newpage
    \section{Exercises}
    
\subsection{Symmetric Polynomial}
    \begin{ex}{14.10}
        Express the symmetric polynomials $\sum_n T_1^2T_2$ and $\sum_{n} T_1^3T_2$ in the elementary symmetric polynomials.
    \end{ex}
    \begin{sol}
        To get the polynomial $\sum_n T_1^2T_2$ we start with
        $$
            s_1s_2=\sum_n T_1\sum_n T_1T_2 = \sum_n T_1^2T_2+3\sum_n T_1T_2T_3 = \sum_n T_1^2T_2+3s_3
        $$
        Thus 
        $$
            \sum_n T_1^2T_2 = s_1s_2-3s_3
        $$

        Similarly, to transform the polynomial $\sum_{n} T_1^3T_2$ we start with
        \begin{align*}
            s_1^2s_2&=\left(\sum_nT_1\right)^2\sum_nT_1T_2\\
            &=\left(\sum_n T_1^2+2\sum_n T_1T_2\right)\sum_nT_1T_2\\
            &=\sum_nT_1^2\sum_n T_1T_2+2s_2^2\\
            &=\sum_nT_1^3T_2+\sum_n T_1^2T_2T_3+2s_2^2.
        \end{align*}
        And since
        $$
            s_1s_3=\sum_nT_1\sum_nT_1T_2T_3=\sum_nT_1^2T_2T_3+4\sum_n T_1T_2T_3T_4
        $$
        it follows that $\sum_n T_1^2T_2T_3=s_1s_3-4s_4$ and so
        $$
            \sum_{n} T_1^3T_2=s_1^2s_2-s_1s_3+4s_4-2s_2^2
        $$
    \end{sol}

    \begin{ex}{14.14}
        Prove: For $n\in\Z_{>0}$, we have $\Delta(X^n+a)=(-1)^{\frac12n(n-1)}n^na^{n-1}$.
    \end{ex}
    \begin{proof}
        Let $f(X)=X^n+a$ and let $\alpha_i$ be its roots. Then $f'(X)=nX^{n-1}$ and
        $$
            \Delta(f)=(-1)^{n(n-1)/2}R(f,f').
        $$
        Let $f_1(X)=a$ and then $f\equiv f_1\mod(f')$ since $f = f_1+f'\cdot\left(\frac1n X\right)$.
        Simplifying the resultant we get
        \begin{align*}
            R(f,f')&=R(f',f)&&(\text{Property }1)\\
            &=n^{n}R(f',f_1)&&(\text{Property }3)\\
            &=n^{n}\cdot\left(n^0\prod_{i=1}^{n-1}f_1(\alpha_i)\right)&&(\text{Property }2)\\
            &=n^n a^{n-1}
        \end{align*}
        and the result follows.
    \end{proof}

    \begin{ex}{14.15}
        Calculate the discriminant of the polynomial $f(X)=X^4+pX+q\in\Q(p,q)[X]$.
    \end{ex}
    \begin{sol}
        Then $f'(X)=4X^3+p$ and so 
        $$f_1(X)=f-f'\cdot h = X^4+pX+q+(4X^3+p)(\frac14 X) = \frac{3p}{4}X+q.$$
        Then the resultant is
        \begin{align*}
            R(f,f')&=R(f',f)&&(\text{Property } 1)\\
            &=4^{4-1}R(f', f_1)&&(\text{Property } 3)\\
            &=4^3\left((-1)^{3\cdot 1}R(f_1,f')\right)&&(\text{Property } 1)\\
            &=-4^3\left(\left(\frac{3p}{4}\right)^3\prod_{i=1}^{1}f'\left(\frac{-4q}{3p}\right)\right)&&(\text{Property } 2)\\
            &=-3^3p^3\left(4\left(\frac{-4q}{3p}\right)^3+p\right)\\
            &=4^4q^3-3^3p^4.
        \end{align*}
        Therefore the discriminant of $f$ is
        $$
            \Delta(f) = (-1)^{4\cdot 3/2}R(f,f') = R(f, f') = 4^4q^3-3^3p^4.
        $$
    \end{sol}

    \begin{ex}{14.16}
        For every $n>1$, determine an expression for the discriminant of the polynomial $f(X) = X^n+pX+q\in\Q(p,q)[X]$.
    \end{ex}
    \begin{sol}
        Let $f(X)=X^n+pX+q\in\Q(p,q)[X]$ for $n>1$. 
        Then $f'(X)=nX^{n-1}+p$ and $f\equiv f_1\mod(f')$ where
        $$f_1 = f-f'\cdot h = X^n+pX+q-\left(nX^{n-1}+p\right)\left(\frac1n X\right)=\frac{p(n-1)}{n}X+q.$$
        The resultant of $f$ and $f'$ is given by
        \begin{align*}
            R(f,f') &= R(f', f)&&(\text{Property } 1)\\
            &=n^{n-1}R(f', f_1)&&(\text{Property } 3)\\
            &=n^{n-1}\left((-1)^{n-1}R(f_1, f')\right)&&(\text{Property } 1)\\
            &=(-n)^{n-1}\left(\frac{p(n-1)}{n}\right)^{n-1}\prod_{i=1}^1 f'\left(-\frac{nq}{(n-1)p}\right)&&(\text{Property } 2)\\
            &=(-1)^{n-1}p^{n-1}(n-1)^{n-1}\left(\frac{(-1)^{n-1}n^nq^{n-1}}{(n-1)^{n-1}p^{n-1}}+p\right)\\
            &=n^nq^{n-1}+(-1)^{n-1}p^n(n-1)^{n-1}.
        \end{align*}
        Hence the discriminant of $f$ is
        $$
            \Delta(f)=(-1)^{n(n-1)/2}R(f,f')=(-1)^{n(n-1)/2}\left(n^nq^{n-1}+(-1)^{n-1}p^n(n-1)^{n-1}\right)
        $$
    \end{sol}

    \begin{ex}{14.17}
        Let $f\in\Z[X]$ be a monic polynomial. Prove that the following are equivalent
        \begin{enumerate}
            \item $\Delta(f)\neq 0$.
            \item The polynomial $f$ has no double zeroes in $\C$.
            \item The decomposition of $f$ in $\Q[X]$ has no multiple prime factors.
            \item The polynomial $f$ and its derivative $f'$ are relatively prime in $\Q[X]$.
            \item The polynomial $f\mod p $ and $f' \mod p$ are relatively prime in $\mathbb{F}_p[X]$ for almost all prime numbers $p$.
        \end{enumerate}
    \end{ex}
    \begin{proof}
        Let  $f\in\Z[X]$ be monic and $\{\alpha_1,\alpha_2,\dots,\alpha_n\}$ it roots in $\C$.

        $(1)\Rightarrow  (2)$. Suppose that $\alpha_i=\alpha_j$ for some $i\neq j$. Then 
        $$\Delta(f)=\prod_{1\leq i<j\leq n}(\alpha_i-\alpha_j)= 0,$$ 
        which is a contradiction.
        Therefore if $f$ has non-zero discriminant it has no double zeroes in $\C$. 

        $(2)\Rightarrow (3)$.
        
        $(3)\Rightarrow (4)$.

        $(4)\Rightarrow (5)$. If $f$ and $f'$ are relatively prime in $\Q[X]$ then 

        $(1)\Rightarrow (1)$. 
    \end{proof}

    \begin{ex}{14.19}
        Let $f\in\Q[X]$ be a monic polynomial with $n=\deg(f)$ distinct complex roots. Prove: the sign of $\Delta(f)$ is equal to $(-1)^s$ where $2s$ is the number of non-real zeroes of $f$.
    \end{ex}
    \begin{proof}
        Let $\{\alpha_1,\dots,\alpha_{n}\}$ be all the roots of $f$.
        Then each term $(\alpha_i-\alpha_j)^2$ in the discriminant falls into one of 3 cases
        \begin{enumerate}
            \item Both $\alpha_i$ and $\alpha_j$ are non-real. Then
            \begin{enumerate}
                \item If $\alpha_j=\overline{\alpha_i}$ then $\alpha_i-\alpha_j$ is purely complex and $(\alpha_i-\alpha_j)^2$ is negative.
                \item If $\alpha_j\neq\overline{\alpha_i}$ then $\overline{\alpha_i}$ and $\overline{\alpha_j}$ are also roots of $f$ and the term
                $$(\alpha_i-\alpha_j)^2(\overline{\alpha_i}-\overline{\alpha_j})^2=\left((\overline{\alpha_i-\alpha_j})(\alpha_i-\alpha_j)\right)^2=\abs{\alpha_i-\alpha_j}^2 $$
                is positive.
            \end{enumerate}
            \item $\alpha_i$ is non-real and $\alpha_j$ is real. Then $\overline{\alpha_i}$ is a root of $f$ and the term
            $$(\alpha_i-\alpha_j)^2(\overline{\alpha_i}-\alpha_j)^2=\abs{\alpha_i-\alpha_j}^2 $$
            is positive.
            \item Both $\alpha_i$ and $\alpha_j$ are real. Then $(\alpha_i-\alpha_j)^2$ is positive.
        \end{enumerate}
        Since the only negative terms are of the form $(\alpha_i-\overline{\alpha_i})^2$ and there are $2s$ non-real roots the sign of the determinant is $(-1)^s$.

    \end{proof}

    \begin{ex}{14.20}
        Prove: $f(X)=X^3+pX+q\in\R[X]$ has three (counted with multiplicity) real zeroes $\iff$ $4p^3+27q^\leq 0$.
    \end{ex}
    \begin{proof}
        By Ex. 16 we know that $\Delta(f)=(-1)^3\left(3^3q^2+2^2p^3\right)=-27q^2-4p^3$. 
        Let $a,b$ and $c$ be the roots of $f$. If $a,b,c\in\R$ then 
        $$
        -27q^2-4p^3=\Delta(f)=(a-b)^2(a-c)^2(b-c)^2\geq 0
        $$
        and so $4p^3+27q^\leq 0$.

        Now suppose that $a=x+yi$ and $b=x-yi$ are complex conjugates and $c$ is real. Then 
        \begin{align*}
            -27q^2-4p^3&=\Delta(f)\\
            &=(a-b)^2(a-c)^2(b-c)^2\\
            &=-4y^2\left((a-c)(\overline{a-c})\right)^2\\
            &=-4y^2\abs{a-c}^2\\
            &\leq 0.
        \end{align*}
        Hence $4p^3+27q^\geq 0$ and the result follows by contraposition.
    \end{proof}
        
    \begin{ex}{14.21}
        Express $p_4=\sum_nT_1^4$ in elementary symmetric polynomials
    \end{ex}
    \begin{sol}
        Let $n\geq 4$. Starting with
        \begin{align*}
            s_1^4 &= \left(\sum_nT_1\right)^4\\& = \sum_n T_1^4+4\sum_n T_1^3T_2+12\sum_n T_1^2T_2T_3+6\sum_nT_1^2T_2^2+24\sum_nT_1T_2T_3T_4.
        \end{align*}
        To understand how to coefficients of the sum are obtained, consider the number of ways the $T_i$ can be arranged. 
        For example, $T_1^4=T_1T_1T_1T_1$ can only be arranged in 1 way but $T_1^2T_2T_3=T_1T_1T_2T_3$ can be arrange in $\frac{4!}{2}=12$ ways (where we divided by 2 since the two $T_1$ can be swapped in any given arrangement).
        Then
        $$
            s_1^2s_2=\left(\sum_n T_1\right)^2s_2=\left(\sum_nT_1^2+2\sum_n T_1T_2\right)s_2 = \sum_n T_1^3T_2+\sum_nT_1^2T_2T_3+2s_2^2.
        $$
        So far we have
        \begin{align*}
            p_4 &= s_1^4-4\left(s_1^2s_2-2s_2^2-\sum_nT_1^2T_2T_3\right)-12\sum_n T_1^2T_2T_3-6\sum_nT_1^2T_2^2-24\sum_nT_1T_2T_3T_4\\
            &=s_1^4-4s_1^2s_2+8s_2^2-24s_4-6\sum_nT_1^2T_2^2-8\sum_n T_1^2T_2T_3.
        \end{align*}
        So continuing with $\sum_nT_1^2T_2^2$ we get
        $$
            s_2^2 = \left(\sum_n T_1T_2\right)^2=\sum_n T_1^2T_2^2+2\sum_n T_1^2 T_2T_3+6\sum_n T_1T_2T_3T_4.
        $$
        Finding the coefficients here is slightly trickier since $s_2$ contains pairs not all arrangements are allowed. 
        For example, $T_1^2T_2^2$ can only come from the pair $T_1T_2$. On the other hand $T_1T_2T_3T_4$ can come from $T_1T_2$ and $T_3T_4$ or $T_1T_4$ and $T_2T_3$ and so on.
        We choose the first pair (${4\choose 2}=6$ ways) which also fixes the second pair and so there are 6 ways to get $T_1T_2T_3T_4$.
        Hence
        \begin{align*}
            p_4 &= s_1^4-4s_1^2s_2+8s_2^2-24s_4-6\left(s_2^2-2\sum_nT_1^2T_2T_3-6s_4\right)-8\sum_n T_1^2T_2T_3\\
            &=s_1^4-4s_1^2s_2+2s_2^2+12s_4+4\sum_n T_1^2T_2T_3.
        \end{align*}
        Using Exercise 14.10 we get
        \begin{align*}
            p_4 &=s_1^4-4s_1^2s_2+2s_2^2+12s_4+4(s_1s_3-4s_4)\\
            &=s_1^4-4s_1^2s_2+2s_2^2-4s_4+4s_1s_3
        \end{align*}
    \end{sol}

    \begin{ex}{14.22}
        A rational function $f\in\Q[T_1,\dots,T_n]$ is called symmetric if it is invariant under all permutations of the variables $T_i$. Prove that every symmetric rational function is a rational function in the elementary symmetric functions.
    \end{ex}
    \begin{proof}
        Let $f\in\Q[T_1,\dots,T_n]$ be a symmetric rational function. 
        Then $f=g/h$ for $g,h$ polynomials. If $h$ is a symmetric polynomial then $g=fh$ is symmetric as well.
        By the fundamental theorem of symmetric polynomial both $g$ and $h$ can be written in terms of elementary symmetric polynomials and we're done.
        If $h$ is not symmetric, then let 
        $$\tilde{h}=\prod_{\sigma\in S_n\setminus\{e\}}\sigma(h)$$
        and then $h\tilde{h}$ is symmetric so $f=\frac{g\tilde{h}}{h\tilde{h}}$ which is again the case above.
    \end{proof}

    \begin{ex}{14.23}
        Write $\sum_{n}T_1^{-1}$ and $\sum_n T_1^{-2}$ as rational functions in $\Q[s_1,\dots,s_n]$
    \end{ex}
    \begin{sol}
        Starting with
        $$
            \sum_{n}T_1^{-1}=\frac{1}{T_1}+\cdots+\frac{1}{T_n}.
        $$
        We multiply by $1=\frac{s_n}{s_n}$ and simplify
        \begin{align*}
            \frac{s_n}{s_n}\sum_{n}T_1^{-1}&=\frac{T_1T_2\cdots T_n}{T_1T_2\cdots T_n}\left(\frac{1}{T_1}+\cdots+\frac{1}{T_n}\right)\\
            &=\frac{s_{n-1}}{s_n}
        \end{align*}

        For the second expression we present to approaches.
        \begin{enumerate}
            \item Observing that 
                $$\left(\sum_n T_1^{-1}\right)^2=\sum_{n} T_1^{-2}+2\sum_{n}T_1^{-1}T_2^{-1}$$
            we can write using the previous part
                $$ \sum_n T_1^{-2} = \frac{s_{n-1}^2}{s_n^2}-2\sum_{n}T_1^{-1}T_2^{-1}$$
            and multiplying by the second term by $\frac{s_{n}}{s_{n}}$ we get
                $$ \sum_n T_1^{-2} = \frac{s_{n-1}^2}{s_n^2} - 2\left(\frac{1}{T_1T_2}+\cdots+\frac{1}{T_{n-1}T_n}\right)\frac{T_1\cdots T_n}{T_1\cdots T_n}=\frac{s_{n-1}^2}{s_n^2} - 2\frac{s_{n-2}}{s_n}.$$
            Hence $\sum_n T_1^{-2}=\frac{s_{n-1}^2-2s_{n-2}s_n}{s_n^2}$.
            \item The second approach is slightly more involved. We start by multiplying by 1 in a clever (but different) way
                $$\left(\sum_n T_1^{-2}\right)\frac{s_n^2}{s_n^2}=\left(\frac{1}{T_1^2}+\cdots+\frac{1}{T_n^2}\right)\frac{T_1^2\cdots T_n^2}{T_1^2\cdots T_n^2}=\frac{\sum_n T_1^2\cdots T_{n-1}^2}{s_n^2}.$$
            Then $\sum_n T_1^2\cdots T_{n-1}^2$ is obviously (condescending much?) a symmetric polynomial and so we can use our trusty algorithm. Starting with
            \begin{align*}
                s_1^{2-2}s_2^{2-2}\cdots s_{n-1}^{2-0}&=s_{n-1}^2\\
                &=\left(\sum_n T_1\cdots T_{n-1}\right)^2\\
                &=\sum_n T_1^2\cdots T_{n-1}^2 + 2\sum_n T_1^2\cdots T_{n-2}^2T_{n-1}T_n.
            \end{align*}
            Moving to the second term
            \begin{align*}
                s_1^{2-2}\cdots s_{n-2}^{2-1}s_{n-1}^{1-1}s_n^1&=s_{n-2}s_n\\
                &=\left(\sum_n T_1\cdots T_{n-2}\right)T_1\cdots T_n\\
                &=\sum_n T_1^2\cdots T_{n-2}^2 T_{n-1}T_n
            \end{align*}
            and it follows that
            $$\sum_n T_1^2\cdots T_{n-1}^2 = s_{n-1}^2-2s_{n-2}s_n.$$
            So we conclude that
            $$ \sum_n T_1^{-2} = \frac{s_{n-1}^2-2s_{n-2}s_n}{s_n^2}$$
            which is reassuring.
        \end{enumerate}
        Note that in the first approach we stumbled upon something rather interesting:
        $$
            \sum_n T_1^{-1}\cdots T_k^{-1} = \frac{s_{n-k}}{s_n}
        $$
        the proof of which is left as an exercise to the reader.
    \end{sol}

    \begin{ex}{14.24}
        
    \end{ex}

\subsection{Field Extensions}
    \begin{ex}{21.18}
        Let $K\subset L$ be an algebraic extension. For $\alpha, \beta\in L$ prove that we have
        $$ \left[K(\alpha,\beta):K\right]\leq\left[K(\alpha):K\right]\cdot\left[K(\beta):K\right]$$.
    \end{ex}
    \begin{proof}
        Let $f$ and $g$ be the minimal polynomials of $\al$ and $\be$ (respectively) in $K[x]$ and $f'$ be the minimal polynomial of $\alpha$ in $K(\beta)[x]$.
        If $\deg f'> \deg f$ then $f$ is a lower degree polynomial in $K(\beta)[x]$ with $f(\alpha)=0$ which is a contradiction. Hence $\deg f'\leq \deg f$ and so
        \begin{align*}
            \left[K(\alpha,\beta):K\right]&=\left[K(\al, \be):K(\be)\right]\cdot\left[K(\be):K\right]\\
            &=\deg f'\cdot \deg g\\
            &\leq\deg f\cdot \deg g\\
            &=\left[K(\al):K\right]\cdot \left[K(\be):K\right],   
        \end{align*}
        as desired.
    \end{proof}

    \begin{ex}{21.19}
        Let $K\subset K(\al)$ be an extension of odd degree. Prove that $K(\al^2)=K(\al)$.
    \end{ex}
    \begin{proof}
        Let $f$ be the minimal polynomial of $\al$ in $K[x]$. Then $\deg f=2n+1$ for some $n\in\Z_+$. 
        Since $\al^2\in K(\al)$ we get the tower $K(\al)/K\left(\al^2\right)/K$ and so\
        $$ \left[K(\al):K\right]=\left[K(\al):K\left(\al^2\right)\right]\cdot\left[K\left(\al^2\right):K\right].$$
        Let $g$ be the minimal polynomial of $\al$ in $K\left(\alpha^2\right)$. Then $\deg g\leq 2$ since $x^2-\al^2\in K\left(\alpha^2\right)$ is a polynomial with a root $\al$.
        Since $\left[K(\al):K\right]$ is odd, it is not divisible by two and so $\deg g = 1$. Hence $\left[K(\al):K\left(\al^2\right)\right]=1$ and it follows that $K(\al)=K\left(\al^2\right)$.
    \end{proof}

    \begin{ex}{21.23}
        Show that every quadratic extension of $\Q$ is of the form $\Q\left(\sqrt{d}\right)$ with $d\in\Z$.
        For what $d$ do we obtain the cyclotomic field $\Q(\zeta_3)$?
    \end{ex}
    \begin{proof}
        Let $K/\Q$ be a quadratic extension. Take $\al\in K\setminus\Q$. Then 
        $$ \Q\subset\Q(\al)\subset K $$
        and so
        $$2=\left[K:\Q\right]=\left[K:\Q(\al)\right]\left[\Q(\al):\Q\right].$$
        If $\left[\Q(\al):\Q\right]=1$ then $\Q(\al)=\Q$ and so $\al\in\Q$, which contradicts our assumption. 
        It follows that $\left[K:\Q(\al)\right]=1$ and so $K=\Q(\al)$. 
        Let 
        $$f(x)=x^2+a_1 x+a_0\in\Q[x]$$
        be the minimal polynomial of $\al$. 
        Let $d=\frac{a_1^2}{4}-a_0\in\Q$ and note that $a_0=-\al a_1-\al^2$. Then
        \begin{align*}
            \sqrt{d}&=\sqrt{\frac{a_1^2}{4}-a_0}\\
            &=\sqrt{\frac{a_1^2}{4}+a_1\al+\al^2}\\
            &=\frac{a_1+2\al}{2}.
        \end{align*}
        Hence $\sqrt{d}\in\Q(\al)$. By similar calculations we get $\al=\frac{2\sqrt{d}-a_1}{2}\in\Q(\sqrt{d})$.
        Hence $K=\Q(\al)=\Q(\sqrt{d})$. Of course, it is not yet the case the $d$ is an integer.
        Suppose that $d=\frac{p}{q}$. Since $\sqrt{d}=\frac{1}{q^2}\sqrt{qp}\in\Q(\sqrt{qp})$ we have
        $$K=\Q(\al)=\Q(\sqrt{d})=\Q(\sqrt{qp})$$
        with $qp\in\Z$ as desired.
    \end{proof}

    \begin{ex}{21.24}
        Is every cubic extension of $\Q$ of the form $\Q\left(\sqrt[3]{d}\right)$ for some $d\in\Q$?
    \end{ex}
    \begin{sol}
        No.
    \end{sol}

    \begin{ex}{21.26}
        Let $M=\Q(\al)=\Q(1+\sqrt{2}+\sqrt{3})$. Show that $M$ is of degree 4 over $\Q$, determine the minimal polynomial and write $\sqrt{2}$ and $\sqrt{3}$ in the basis $\{1,\al, \al^2,\al^3\}$.
        Also prove that the group $G=\text{Aut}_\Q(M)$ is isomorphic to $V_4$ and that $f^\al_\Q=\prod_{\sigma\in G}X-\sigma(\al)\in\Q[X]$.
    \end{ex}
    \begin{sol}
        Let $\be = \al-1=\sqrt{2}+\sqrt{3}$. Then clearly $M=\Q(\al)=\Q(\be)$. Let
        \begin{align*}
            f(x)&=(x-\sqrt{2}-\sqrt{3})(x+\sqrt{2}-\sqrt{3})(x-\sqrt{2}+\sqrt{3})(x+\sqrt{2}+\sqrt{3})\\
            &=x^4-10x^2+1\in\Q[x]
        \end{align*}
        and so $f(\be)=0$ by construction. 
        
        Is $f$ the minimal polynomial of $\be$ in $\Q[x]$? It is if we can prove that $[M:\Q]=4$.
        From
        $$ (\sqrt{2}+\sqrt{3})(\sqrt{3}-\sqrt{2})=1 $$
        It follows that $\be^{-1}=\sqrt{3}-\sqrt{2}$. Therefore
        $$ \sqrt{2}=\frac12(\be-\be^{-1})\quad\text{and}\quad\sqrt{3}=\frac12(\be+\be^{-1})$$
        and so $M=\Q(\sqrt{2}+\sqrt{3})=\Q(\sqrt{2},\sqrt{3})$. 
        Hence we have the towers $M/\Q(\sqrt{2})/Q$ and $M/\Q(\sqrt{3})/Q$. 
        Let $g(x)=x^2-3$. Suppose it is not the minimal polynomial of $\sqrt{3}$ in $\Q(\sqrt{2})$.
        Then there exists $a+b\sqrt{2}\in\Q(\sqrt{2})$ such that
        $$ 0 = g(a+b\sqrt{2})=a^2+2b^2-3+2ab\sqrt{2}.$$
        But since
        \begin{equation*}
            \begin{cases}
                a^2+2b^2-3=0\\
                2ab=0
            \end{cases}
        \end{equation*}
        has no solutions it follows that no such element exists.
        Therefore $g$ is the minimal polynomial of $\sqrt{3}$ and $[M:\Q(\sqrt{2})]=\deg g=2$.
        Since $x^2-2$ is the minimal polynomial of $\sqrt{2}$ in $\Q$ we conclude that 
        $$[M:\Q]=[M:\Q(\sqrt{2})]\cdot[\Q(\sqrt{2}):\Q)]=4$$ 
        and therefore $f$ is the minimal polynomial of $\be$.

        Thus $f(x-1)$ is the minimal polynomial of $\al$ in $\Q$. 
        From $f(\be)=0$ it follows that $1=\beta(10\beta-\beta^3)$ and so $\be^{-1}=10\beta-\beta^3$.
        Hence
        $$\sqrt{2}=\frac12\left(\be-\be^{-1}\right)=\frac12\left(\be-10\be+\be^3\right)=\frac12\left(-9(\al-1)+(\al-1)^3\right)$$
        and
        $$\sqrt{3}=\frac12\left(\be+\be^{-1}\right)=\frac12\left(11(\al-1)-(\al-1)^3\right)$$

        Let $G=\text{Aut}(M)$ and take $\sigma\in G$. Then by definition $\sigma(1)=1$ and it follows by induction and the properties of isomorphism that $\sigma(a)=a$ for all $a\in\Z$.
        Since $1=\sigma(1)=\sigma(a\cdot a^{-1})=\sigma(a)\cdot\sigma(a)^{-1}=a\cdot a^{-1}$ it also follows that $\sigma\left(\frac{p}{q}\right)=\frac{p}{q}$. 
        Hence $\sigma$ restricted to $\Q$ is simply the identity map. 
        Therefore $\sigma$ is completely determined by $\sigma(\sqrt{2})$ and $\sigma(\sqrt{3})$.
        Since $0=\sigma(0)=\sigma(\sqrt{2}^2-2)=\sigma(\sqrt{2})^2-2$ the only options are $\sigma(\sqrt{2})=\pm\sqrt{2}$.
        Similarly we conclude that $\sigma(\sqrt{3})=\pm\sqrt{3}$. This gives four possible automorphism.
        Take $\sigma,\tau\in G$ such that $\sigma(\sqrt{2})=-\sqrt{2}, \sigma(x)=x$ $\forall x\in M\setminus\{\sqrt{2}\}$ and $\tau(\sqrt{3})=-\sqrt{3},\tau(x)=x$ $\forall x\in M\setminus\{\sqrt{3}\}$. 
        Since 
        $$\sigma\circ\sigma=\tau\circ\tau=\sigma\circ\tau\circ\sigma\circ\tau=e$$
        where $e$ is the identity map it follows that $G$ is isomorphic to $V_4$, the Klein four-group.

        Lastly, consider
        \begin{align*}
            \tilde{f}&=\prod_{\sigma\in G}x-\sigma(\al)\\
            &=(x-1-\sqrt{2}-\sqrt{3})(x-1+\sqrt{2}-\sqrt{3})(x-1-\sqrt{2}+\sqrt{3})\\&\qquad\qquad (x-1+\sqrt{2}+\sqrt{3}).
        \end{align*}
        Hence $\tilde{f}(x)=f(x-1)$ which we already proved is the minimal polynomial of $\al$ in $\Q[x]$.

    \end{sol}

    \begin{ex}{21.29}
        Take $K=\Q(\al)$ with $f^\al_\Q=x^3+2x^2+1$.
        \begin{enumerate}
            \item Determine the inverse of $\al+1$ in the basis $\{1,\al,\al^2\}$ of $K$ over $\Q$.
            \item Determine the minimal polynomial of $\al^2$ over $\Q$.
        \end{enumerate}
    \end{ex}

    \begin{ex}{21.30}
        Define the cyclotomic field $\Q(\zeta_3)$ and write $\al=\zeta_5^2+\zeta_5^3$.
        \begin{enumerate}
            \item Show that $\Q(\al)$ is a quadratic extension of $\Q$ and determine $f^\al_\Q$.
            \item Prove: $\Q(\al)=\Q(\sqrt{5})$
            \item Prove: $\cos(2\pi/5)=\frac{\sqrt{5}-1}{4}$ and $\sin(2\pi/5)=\sqrt{\frac{5+\sqrt{5}}{8}}$
        \end{enumerate}
    \end{ex}
\subsection{Finite Fields}

\subsection{Separable and Normal Extensions}
\end{document}

