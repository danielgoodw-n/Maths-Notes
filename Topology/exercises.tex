\subsection{Basis for a topology}
\begin{ex}{13.1}
    Let $X$ be a topological space; let $A\subset X$. Suppose that for each $x\in A$ there is an open set $U$ such that $x\in U\subset A$. Show that $A$ is open.
\end{ex}
\begin{proof}
    For every $x\in A$, let $U_x$ denote the open set containing $x$ such that $U_x\subset A$. Then $U = \bigcap_{x\in A} U_x\subset A$ since each $U_x$ is contained in $A$.
    For the other inclusion, take $x\in A$. Then $x\in U$ since $x$ is in $U_x$ by definition. Hence $A\subset U$ and it follows that $A=U$. 
    Since each $U_x$ and arbitrary unions of open sets are open it follows that $A$ is open.     
\end{proof}

\begin{ex}{13.4}
    ${}$
    \begin{enumerate}
        \item If $\{\Ta_\alpha\}$ is a family of topologies on $X$,how that $\bigcap\Ta_\alpha$ is a topology on $X$. Is $\bigcup\Ta_\alpha$ a topology on $X$?
        \item Let $\{\Ta_\alpha\}$ be a family of topologies on $X$. Show that there is a unique smallest topology containing all the all the collection of $\Ta_\alpha$ and a unique largest topology contained in all $\Ta_\alpha$
        \item If $X=\{a,b,c\}$ let
            $$
                \Ta_1 = \{\varnothing, X, \{a\}, \{a,b\}\}\quad\text{and}\quad\Ta_2=\{\varnothing, X,\{a\}, \{b,c\}\}.
            $$
            Find the smallest topology containing $\Ta_1$ and $\Ta_2$ and the largest topology contained in $\Ta_1$ and $\Ta_2$.
    \end{enumerate}
\end{ex}
\begin{sol}
    ${}$
    \begin{enumerate}
        \item (a) Since $\varnothing, X\in\Ta_\alpha$ for all $\alpha$ it follows that $\varnothing, X\in\bigcap\Ta_\alpha$. (b) If $U_\beta\in\bigcap\Ta_\alpha$, then $U_\beta\in\Ta_\alpha$ for all $\alpha$ and so $\bigcup U_\beta\in\Ta_\alpha$ for all $\alpha$ since $\Ta_\alpha$ is a topology. Hence $\bigcup U_\beta\in\bigcap\Ta_\alpha$. 
        (c) If $U_1,U_2\in\bigcap\Ta_\alpha$ then $U_1,U_2\in\Ta_\alpha$ for all $\alpha$ and so $U_1\cap U_2\in\Ta_\alpha$ for all $\alpha$. Therefore $U_1\cap U_2\in\bigcap \Ta_\alpha$. It follows by induction that $\bigcap\Ta_\alpha$ is closed under countable intersections. Hence an intersections of topologies is a topology.
        
        Let $X=\{a,b,c\}$. Then $\Ta_1=\{\varnothing, X, \{a\}\}$ and $\Ta_2=\{\varnothing, X, \{b\}\}$ are topologies on $X$. But $\Ta_1\cup\Ta_2=\{\varnothing, X, \{a\}, \{b\}\}$ is not a topology since $\{a\},\{b\}\in\Ta_1\cup\Ta_2$ but $\{a\}\cup\{b\}=\{a,b\}\not\in\Ta_1\cup\Ta_2$. Hence a union of topologies is, in general, not a topology. 
        
        \item Let $\Ss=\bigcup_\alpha \Ta_\alpha$. Then $X\in\Ss$ since $X$ is in each individual $\Ta_\alpha$ as they are all topologies. It follows that $X=\bigcup_{S\in\Ss} S$ and so $\Ss$ is a subbasis. 
        Let $\Ba$ be the basis generated by $\Ss$ and $\Ta_s$ be the topology generated by $\Ba$. Fix some $\Ta_\alpha$ and take $U\in\Ta_\alpha$. Then $U\in\Ss\subset\Ba\subset\Ta_\Ss$ by construction.
        Hence $U\in\Ta_\Ss$ and it follows that $\Ta_\alpha\subset\Ta_\Ss$ for all $\alpha$. Is it the smallest topology with such property? Let $\Ta'$ be a topology on $X$ such that $\Ta_\alpha\subset\Ta'$ for all $\alpha$ and take $U\in\Ta_\Ss$.
        Then $U$ is an arbitrary union of finite intersections of elements of $\Ss=\bigcup\Ta_\alpha\subset\Ta'$. Since $\Ta'$ is a topology it is closed under arbitrary unions and finite intersections and so $U\in\Ta'$. Hence $\Ta_\Ss\subset\Ta'$ and it follows that $\Ta_\Ss$ is the smallest topology containing all $\Ta_\alpha$.

        From part one we know that $\bigcap\Ta_\alpha$ is a topology, and by definition it is contained in $\Ta_\alpha$ for all $\alpha$. 
        If $\Ta'\subset\Ta_\alpha,\,\forall\alpha$ is a topology, then for every $U\in\Ta'$, $U\in\bigcap\Ta_\alpha$ and so $\Ta'\subset\bigcap\Ta_\alpha$. Therefore $\bigcap\Ta_\alpha$ is the largest topology that is contained in all $\Ta_\alpha$.
        
        \item Apply part (2).
    \end{enumerate}
\end{sol}

\begin{ex}{13.5}
    Show that that topology $\Ta$ on $X$ generated by a basis $\Ba$ is equal to the intersections of all the topologies on $X$ that contain $\Ba$.
\end{ex}
\begin{proof}
    Let $T = \left\{\Ta_{\beta}\mid \Ba\subset\Ta_{\beta}\right\}$ be the collection of all topologies on $X$ that contain $\Ba$. Let $u\in\Ta$. Then $U$ can be written as a union of element in $\Ba$, i.e.
    $$
        U = \bigcup_{\alpha} B_\alpha,\quad B_\alpha\in\Ba
    $$
    Since $\Ta_\beta$ is a topology and $\Ba\subset\Ta_\beta$ for all $\Ta_\beta\in T$ it follows that $U= \bigcup_{\alpha} B_\alpha\in\Ta_\beta$ for all $\Ta_\beta\in T$ and so 
    $$\Ta\subset\bigcap_{\Ta_\beta\in T}\Ta_\beta.$$

    Since $\Ba\subset\Ta$ by definition of a basis it follows that $\Ta\in T$ and so 
    $$
    \bigcap_{\Ta_\beta\in T}\Ta_\beta\subset\Ta.
    $$
    Hence $\bigcap_{\Ta_\beta\in T}\Ta_\beta=\Ta$.
\end{proof}

\begin{ex}{13.7}
\end{ex}
\begin{sol}
    
\end{sol}

\begin{ex}{13.8}
    ${}$
    \begin{enumerate}
        \item Show that the collection 
        $$
            \Ba=\left\{(a,b)\mid a<b, a,b\in\Q \right\} 
        $$
        generates the standard topology on $\R$.
    
        \item Show that the collection
        $$
            \mathcal{C}=\left\{[a,b)\mid a<b, a,b\in\Q\right\} 
        $$
        generates a topology different from the lower limit topology.
    \end{enumerate}
\end{ex}
\begin{sol}
    ${}$
    \begin{enumerate}
        \item Let $\Ta$ be the standard topology on $\R$ and $\Ta_\Ba$ the topology generated by $\Ba$. 
        Let $x\in(a,b)\in\Ta$. Since the rationals are dense in $\R$, there exist $a', b'\in\Q$ such that $x\in(a',b')\subset(a,b)$. 
        Hence $\Ta\subset\Ta_\Ba$. The other inclusion is trivial since every basis element $(a,b)\in\Ba$ is a basis element of $\Ta$. We conclude that $\Ta=\Ta_\Ba$.

        \item Let $\Ta_c$ be the topology generated by $\mathcal{C}$ and let $\Ba$ be the basis of $\Ta_l$, the lower limit topology on $\R$. 
        Then for any $[a,b)\in\mathcal{C}$, $[a,b)\in\Ba$, and so $\Ta_\mathcal{C}\subset\Ta_l$. To show that this inclusion is strict we need to prove the statement
        \begin{align*}
            \lnot&\left(\forall B\in\Ba\,\forall x\in B\,\exists C\in\mathcal{C}: x\in C\subset B\right)\\
            \iff&\exists B\in\Ba\, \exists x\in B\,\forall C\in\mathcal{C}:x\not\in C\vee C\not\subset B\\
            \iff&\exists B\in\Ba\, \exists x\in B\,\forall C\in\mathcal{C}:x\in C\implies C\not\subset B
        \end{align*}
        Let $[x,b)\in\Ba$ with $x\not\in\Q$. Then $[a,c)\in\mathcal{C}$ can contain $x$ only if $a<x$ since $x$ is irrational. Therefore there is no element in $\mathcal{C}$ that contains $x$ and is a subset of $[x,b)$.
        This proves that $\Ta_\mathcal{C}\subsetneq\Ta_l$.
    \end{enumerate}
\end{sol}

\subsection{The Subspace Topology}
\begin{ex}{16.1}
    Show that if $Y$ is a subspace of $X$, and $A$ is a subset of $Y$ then the topology $A$ inherits as a subspace of $Y$ is the same as the topology it inherits as a subspace of $X$.
\end{ex}
\begin{proof}
    Let $\Ta$ be a topology on $X$, $\Ta_Y$ subspace topology on $Y$. Let $\Ta_A'$ be the topology $A$ inherits as a subset of $Y$.
    Then
    \begin{align*}
        \Ta_A'&=\left\{A\cap U\mid U\in\Ta_Y\right\}\\
        &=\big\{A\cap U\mid U\in\left\{Y\cap V\mid V\in \Ta\right\} \big\}\\
        &=\left\{A\cap U\mid U= Y\cap V, V\in\Ta\right\}\\
        &=\left\{A\cap\left(Y\cap V\right)\mid V\in\Ta\right\}\\
        &=\left\{\left(A\cap Y\right)\cap V\mid V\in\Ta\right\}\\
        &=\left\{A\cap V\mid V\in\Ta\right\}    
    \end{align*}
    which is by definition the topology $A$ inherits as a subset of $X$.
\end{proof}

\begin{ex}{16.3}
    
\end{ex}

\begin{ex}{16.4}
    Show that $\pi_1:X\times Y\to X$ is an open map.
\end{ex}
\begin{proof}
    Let $U$ be open in $X\times Y$ and take $(x,y)\in U$. Then there exists a basis element $B_x\times B_y$ such that $(x,y)\in B_x\times B_y\subset U$.
    For any $b\in B_x$, $(b,y)\in B_x\times B_y\subset U$ and so $b=\pi_1(b,y)\in \pi_1(U)$. It follows that $B_x\subset\pi_1(U)$.
    Since the basis of a product topology is the the product of open sets, $B_x$ is open in $X$ which means that for every $x\in\pi_1(U)$ there is an open set $B_x\in X$ such that $x\in B_x\subset \pi_1(U)$.
    From Exercise 13.1 it follows that $\pi_1(U)$ is open in $X$.    
\end{proof}


