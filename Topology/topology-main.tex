\documentclass{article}

\usepackage[utf8]{inputenc}
\usepackage{csquotes}
\usepackage[english]{babel}
\usepackage{amsmath,amssymb,amsthm,textcomp}
\usepackage{mathtools}
\usepackage{biblatex}
\usepackage{tikz}
\usepackage{graphics, setspace}
\usepackage{listings}
\usepackage{lipsum}
\usepackage{hyperref}
\hypersetup{
    colorlinks,
    citecolor=black,
    filecolor=black,
    linkcolor=black,
    urlcolor=black
}

\DeclareMathAlphabet{\pazocal}{OMS}{zplm}{m}{n}
\DeclareMathOperator{\Ima}{Im}
\newcommand{\Ba}{\mathcal{B}}
\newcommand{\Ta}{\mathcal{T}}
\newcommand{\Aa}{\mathcal{A}}
\newcommand{\R}{\mathbb{R}}
\newcommand{\C}{\mathbb{C}}
\newcommand{\Z}{\mathbb{Z}}
\newcommand{\N}{\mathbb{N}}
\newcommand{\p}{\mathbb{P}}
\newcommand{\Ss}{\mathcal{S}} % Schwartz space
\newcommand{\F}{\mathcal{F}} % Fourier Transform
\newcommand{\Rf}{\mathcal{R}} % reflection
\newcommand{\E}{\mathbb{E}}

\DeclarePairedDelimiter\abs{\lvert}{\rvert}%
\DeclarePairedDelimiter\norm{\lVert}{\rVert}%
% Swap the definition of \abs* and \norm*, so that \abs
% and \norm resizes the size of the brackets, and the 
% starred version does not.
\makeatletter
\let\oldabs\abs
\def\abs{\@ifstar{\oldabs}{\oldabs*}}
%
\let\oldnorm\norm
\def\norm{\@ifstar{\oldnorm}{\oldnorm*}}
\makeatother

\newtheorem{theorem}{Theorem}[section]
\newtheorem{corollary}{Corollary}[theorem]
\newtheorem{lemma}[theorem]{Lemma}

\title{Topology - X400416}
\author{Yoav Eshel}
\date{\today}

\begin{document}
\maketitle
\tableofcontents
\newpage
These notes are based on Topology ($2^\text{nd}$ edition) by James R. Munkres. 
\section{Topological Spaces and Continuous Functions}
A metric on a set $X$ is a map $d:X\times X \to [0,\infty)$ that satisfies
\begin{enumerate}
    \item $d(x,y)=d(y,x)$
    \item $d(x,x)=0$
    \item $d(x,y)>0, x\not=y$
    \item $d(x,y)\leq d(x,z)+d(z,y)$
\end{enumerate}
A set $U\subset X$ is open if for all $x\in U$ and some $r>0$ 
$$
    B(x,r) = \{y\in X\mid d(x,y)<r\}
$$
s.t. $B(x,r)\subset U$.
Union (finite, countable or uncountable) of open sets is open and finite intersections of open sets is open (infinite intersections need not be open)

A function between metric spaces is continuous if and only if a preimage of an open set is open. 

\begin{definition}
    Let $X$ be a set. Then a topology on $X$ is a set $\mathcal{T}\subset\mathcal{P}(x)$ s.t.
    \begin{enumerate}
        \item $\varnothing\in\mathcal{J}, X\in\mathcal{J}$
        \item For $U_{\alpha}\in\mathcal{J}, \bigcup_{\alpha}U_{\alpha}\in\mathcal{J}$
        \item For $(U_i)_{0\leq i\leq n}\subset\mathcal{J}, \bigcap_{i=0}^n U_i\in\mathcal{J}$
    \end{enumerate}
    A topological space is the pair $(X,\mathcal{J})$
\end{definition}

If $X$ is a set, the a \textit{basis} of a topology is a collection $\mathcal{B}$ of subsets of $X$ s.t.
\begin{enumerate}
    \item $\forall x\in X, \exists B\in\mathcal{B}$ s.t. $x\in B$
    \item If $x\in B_1\cap B_2$ with $B_1,B_2\in\mathcal{B}$ then there exists $B_3\in\mathcal{B}$ with $x\in B_3\subset B_1\cap B_2$.
\end{enumerate}
    
\end{document}